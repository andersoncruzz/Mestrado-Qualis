\chapter{Abordagem Proposta}\label{sec:abordagemproposta}
Neste capítulo uma abordagem é proposta para reconhecer emoções por meio da expressão facial e está dividido da seguinte forma. Na Seção \ref{sec:detect} descreve um módulo de detecção de face e recorte. Na Seção \ref{sec:preproc} aborda as operações de pré-processamento aplicadas na imagem. Na Seção \ref{sec:redeneu} apresenta o classificador de expressões faciais e por fim um resumo do capítulo na Seção \ref{sec:considfi}.

\section{Detecção de Face e Recorte}\label{sec:detect}
Este procedimento consiste na detecção de todas as faces de uma imagem por meio do algoritmo Viola Jones (consultar Seção \ref{sec:detecfacialviola}) gerando um conjunto de coordenadas para criar um retângulo indicando a localização da face. Vale ressaltar que esta atividade possui complexidade moderada, pois uma imagem contém vários objetos com diferentes geometrias, inclusive podendo assemelhar-se a uma face acarretando na geração de falsos positivos. Logo após a detecção de face é realizado o recorte utilizando o conjunto de coordenadas definido pela etapa anterior, tal atividade é valorosa para exclusão do \textit{background}. Desta forma, é enviado somente a face recortada para a etapa de pré-processamento reduzindo a complexidade do problema, pois não há necessidade do classificador aprender a separar o \textit{background} da face. Posteriormente, ao recorte da face, a imagem original que deve está com uma face recortada é mantida para nova averiguação de recorte de face. Caso exista outras faces na imagem, este processo é repetido até não existir mais faces para recortar. Obviamente caso seja enviada uma imagem para a etapa de detecção e recorte que não contém uma face (e.g. imagem de um avião) o processo é automaticamente encerrado, pois se não há uma face para detectar, logo não há uma expressão facial emocional para reconhecer. 

\section{Pré-Processamento}\label{sec:preproc}
Uma face recortada é enviada pelo módulo de Detecção de Face e Recorte para a fase de Pré-processamento. Nesta etapa, operações de pré-processamento são aplicadas com a finalidade de realçar características relevantes que diferenciam as expressões faciais com intuito de preparar a imagem para a classificação. Inicialmente, uma função de redimensionamento é chamada para transformar a imagem em uma escala de 60x60 \textit{pixels}, como a imagem é colorida, isto é, possui 3 canais denominados RGB (do inglês: Red, Green e Blue) a imagem resultante possui 10.800 características que pode ser calculada por \textit{Qtd_Caracteristica = N_Pixels_X * N_Pixels_Y * N_Canais}. Após o redimensionamento, é realizado a normalização da imagem dividindo cada \textit{pixel} por 255, isto é, o valor máximo que um \textit{pixel} pode possuir, ocorrendo a normalização dos valores dos \textit{pixels} para o intervalo de 0 a 1. 

O problema visualizado por esta proposta consiste em classificar emoções em qualquer ambiente. Obviamente a variação de ambiente acarreta em diferentes níveis de intensidade da luz, ocorrendo a perda de características importantes da face que diferenciam as emoções, seja por excesso ou ausência de luz. Vale destacar que esta proposta é baseada principalmente em redes neurais de convolução que originalmente possui vários filtros de pré-processamento. Entretanto, a literatura tem mostrado que filtros clássicos aplicados antes da inserção de uma imagem em redes neurais tem sido eficazes na eliminação de ruídos, principalmente aqueles relacionados a iluminação e brilho \citep{art2,art4,art6}. Assim, a imagem normalizada por filtros de correção de iluminação ressaltará melhor os traços faciais, além da imagem transformada está com maior nitidez para a rede neural de convolução. Portanto, as técnicas de normalização de brilho e iluminação são parte da etapa de pré-processamento agregando valor para a solução minimizar a sensibilidade relacionada a variação de luz do ambiente.   


\section{Rede Neural de Convolução}\label{sec:redeneu}
A rede neural de convolução é a parte central e mais importante desta abordagem em discussão. Por meio dela a imagem é processada enfatizando contornos, padrões, formas e características da imagem relevantes para a classificação. Além disso, funções são aplicadas para redução de dimensionalidade e normalização ocasionando que a rede não seja sensível a rotações, posições e escala da imagem. Tais aptidões são essenciais para um classificador de imagens que deve ser usado em cenários reais maximizando a generalização. Entretanto, um desempenho satisfatório da rede neural de convolução, assim como de qualquer algoritmo supervisionado de aprendizagem de máquina, está estritamente relacionado ao processo de treinamento e validação do modelo.    

\subsection{Treinamento}
O treinamento da rede neural de convolução é parte fundamental para o reconhecedor de emoção alcançar a generalização satisfatória e adquirir aprendizado suficiente para funcionar em variados ambientes. Para isso, o treinamento é apoiado pela técnica de aumento de dados com intuito de maximizar a generalização do aprendizado durante o treinamento. O aumento de dados consiste na multiplicação das imagens em tempo dinâmico modificando levemente a imagem e seu contexto alterando as imagens aplicando zoom, rotações, \textit{blur}, \textit{shear}, diferentes níveis de contraste e entre outros, estimulando maiores taxas de aprendizado e generalização justamente pela rede neural observar a região de interesse, isto é, as expressões faciais emocionais em diferentes cenários gerando modelos capazes de classificar emoções em contextos variados. 

\subsection{Extração de Características e Classificação}
A rede neural de convolução recebe a imagem pré-processada de uma face para classificá-la estimando a probabilidade para cada emoção: neutralidade, raiva, felicidade, tristeza, desprezo, medo e surpresa, de acordo com as caraterísticas extraídas da imagem.  A extração de característica é um procedimento responsável em identificar as zonas da imagem que são mais relevantes para a separação do problema, isto é, classificar uma expressão facial (\textit{e.g} o sorriso humano é uma expressão facial indicadora para a emoção felicidade). A extração de característica está embutida na rede neural de convolução que consiste nas camadas operando sobre a imagem de entrada ressaltando todos os contornos. Algumas camadas de convolução são especialistas na extração dos padrões verticais, outras nos horizontais, até que um conjunto de características são extraídas para o \textit{softmax} classificar estimando a probabilidade para cada emoção.

%contornos, bordas, linhas, horinzontais, verticais
%Produto de probabilidades Naive Bayes


     

\section{Resumo}\label{sec:considfi}
'%imagens gerais
%LUTE LUTE LUTE LUTE GO FIGHT GO FIGHT YOU CAN ONLY DEPENDS YOU SON OF BEATCH