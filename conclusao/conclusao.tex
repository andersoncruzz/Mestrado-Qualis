\chapter{Considerações Finais}\label{sec:conclusao}
Neste trabalho, foi apresentado uma abordagem para reconhecer emoção por meio da expressão facial utilizando redes neurais de convolução. O diferencial desta abordagem é reunir os principais elementos identificados na literatura para reconhecer emoções. Além disso, este trabalho pretende fornecer uma solução para reconhecer emoções em computação embarcada e em nuvem.     

Uma prova de conceito foi aplicada para verificar a viabilidade do reconhecimento de emoção por expressão facial. Nesta prova de conceito, um serviço em nuvem foi utilizado para reconhecer as emoções. O foco foi monitorar estudantes por meio da captura de imagens em tempo real, enquanto faziam um teste escolar com questões de múltipla escolha. Foram correlacionadas as emoções com o desempenho no teste. Portanto, como resultado principal identificamos que as questões onde ocorreram diferentes emoções, foram as questões que tiveram maiores proporções de acertos.  

Adicionalmente, para gerar nosso próprio reconhecedor de emoção, um estudo experimental foi conduzido com intuito de avaliar três arquiteturas de redes neurais de convolução: AlexNet, Inception-V3 e ResNet-34. Os resultados apontaram que a ResNet-34 obteve as melhores taxas de acurácia, precisão, revocação e f1-score. As imagens utilizadas no estudo foram provenientes da natureza e laboratório. Concluímos que um método treinado pelas imagens da natureza não influenciam negativamente para reconhecer imagens laboratoriais. Em contrapartida, um método treinado com as imagens laboratoriais não consegue reconhecer emoções no contexto da natureza.


\section{Limitações do Trabalho}
Este trabalho tem como objetivo reconhecer emoções humanas por meio da expressão facial. Todavia, os trabalhos de \cite{darwin1965expression} e \cite{ekman1994} apontaram que somente o grupo das emoções básicas (raiva, alegria, tristeza, desgosto, medo e surpresa) são emitidas por meio da expressão facial. Portanto, este trabalho tem possibilidades de reconhecer exclusivamente as emoções básicas. 

\section{Trabalhos Futuros}
\begin{itemize}
 \item \textbf{Analisar sequência de imagens}: Atualmente, a abordagem analisa somente uma imagem para reconhecer as emoções. Desta forma, está sendo ignorada a característica temporal e apenas uma imagem é considerada para definir as emoções. Integrar uma sequência de imagens, na forma de uma série temporal, pode ser relevante para melhorar o reconhecimento, justamente por analisar um conjunto de imagens amostrada em uma fração de tempo para definir as emoções. 
 \item \textbf{Avaliar experimentalmente outros classificadores}: O \textit{softmax} foi o único classificador adotado no estudo realizado. É interessante avaliar o desempenho de outros classificadores analisando diferentes famílias:  SVM (funcional), Gaussiana (probabilística) e RandomForest (ensemble de árvores de decisões).
 \item \textbf{Implementar e avaliar a MobileNet}: A MobileNet é a arquitetura de rede neural de convolução apropriada para sistemas embarcados. É necessário um estudo para parametrizar a MobileNet, mensurar o consumo de recursos computacionais e, além disso, definir as especificações mínimas de hardware para a instalação e funcionamento.  
 \item \textbf{Desenvolver e avaliar o componente de pré-processamento}: As técnicas de alinhamento de face, normalização de iluminação e aumento de dados, devem ser implementadas. É interessante um estudo para avaliar a contribuição e o impacto dessas técnicas. Principalmente, no cenário das imagens oriundas da natureza, em que nesse contexto, o método proposto não obteve bons resultados.
 \item \textbf{Avaliar em cenários de uso reais}: Nos seguintes cenários pretendemos aplicar a solução: na educação ou em tecnologia assistiva para deficientes visuais. Na educação é útil para auxiliar sistemas inteligentes que interagem com aluno de acordo com a emoção do mesmo. Na tecnologia assistiva pode auxiliar deficientes visuais que tem dificuldades em reconhecer emoção, inclusive este tipo de aplicação abre um campo desafiador para a interação humano computador.
 
\end{itemize}


%\chapter{Cronograma}\label{sec:cronograma}

O cronograma está descrito na Tabela \ref{cronog}.

\begin{table}[]\footnotesize
\centering
\caption{Cronograma de Atividades}
\label{cronog}
\begin{tabular}{|A|c|c|c|c|c|c|c|c|c|c|c|c|}
\hline
                                                     & \multicolumn{12}{c|}{\textbf{Ano 2018}}                                                                                                                                           \\ \hline
\textbf{Atividades}                                  & \textbf{jan} & \textbf{fev} & \textbf{mar} & \textbf{abr} & \textbf{mai} & \textbf{jun} & \textbf{jul} & \textbf{ago} & \textbf{set} & \textbf{out} & \textbf{nov} & \textbf{dez} \\ \hline
Implementação do componente de pré-processamento     & x            & x            &              &              &              &              &              &              &              &              &              &              \\ \hline
Implementação da arquitetura GoogLeNet               &              & x            & x            &              &              &              &              &              &              &              &              &              \\ \hline
Implementação da arquitetura VGG                     &              &              & x            &              &              &              &              &              &              &              &              &              \\ \hline
Implementação da arquitetura Ensemble                &              &              & x            & x            &              &              &              &              &              &              &              &              \\ \hline
Download das bases de dados                          & x            & x            &              &              &              &              &              &              &              &              &              &              \\ \hline
Planejamento experimental das RNCs                   &              &              &              & x            &              &              &              &              &              &              &              &              \\ \hline
Treinamento das RNCs                                 &              &              &              &              & x            & x            & x            &              &              &              &              &              \\ \hline
Avaliação experimental das RNCs                      &              &              &              &              &              &              &              & x            &              &              &              &              \\ \hline
Planejamento experimental para cenários de uso reais &              &              &              &              &              &              &              & x            &              &              &              &              \\ \hline
Execução do experimento para cenários de uso reais   &              &              &              &              &              &              &              &              & x            &              &              &              \\ \hline
Avaliação dos resultados experimentais               &              &              &              &              &              &              &              &              &              & x            &              &              \\ \hline
Escrita da dissertação                               &              &              &              &              & x            & x            &              &              & x            & x            & x            & x            \\ \hline
\end{tabular}
\end{table}


%\multicolumn{12}{c|}

\section{Cronograma}\label{sec:cronograma}
O cronograma está descrito na Tabela \ref{table:cronog}.

\begin{table}[ht]\footnotesize
\caption{Cronograma de Atividades}
\label{table:cronog}
\begin{tabular}{l|ccccc|ccc|}
\cline{2-9}
                                                                           & \multicolumn{5}{c|}{\textbf{2018}}                                       & \multicolumn{3}{c|}{\textbf{2019}}         \\ \hline
\multicolumn{1}{|l|}{\textbf{Atividades}}                                  & \textbf{ago} & \textbf{set} & \textbf{out} & \textbf{nov} & \textbf{dez} & \textbf{jan} & \textbf{fev} & \textbf{mar} \\ \hline
\multicolumn{1}{|l|}{Desenvolver e avaliar o componente pré-processamento} & x            &              &              &              &              &              &              &              \\
\multicolumn{1}{|l|}{Analisar sequência de imagens}                        &              & x            &              &              &              &              &              &              \\
\multicolumn{1}{|l|}{Avaliar experimentalmente outros classificadores}     &              &              & x            &              &              &              &              &              \\
\multicolumn{1}{|l|}{Implementar e avaliar a MobileNet}                    &              &              &              & x            &              &              &              &              \\
\multicolumn{1}{|l|}{Avaliar em cenários de uso reais}                     &              &              &              &              & x            & x            &              &              \\
\multicolumn{1}{|l|}{Escrita da dissertação}                                  &              &              &              & x            & x            & x            & x            & x            \\ \hline
\end{tabular}
\end{table}


% 
%\section{Publicações}